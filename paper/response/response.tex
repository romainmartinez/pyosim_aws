% Preamble
\documentclass[preprint,review,12pt]{elsarticle}

% Package
\usepackage{graphicx}
\usepackage{amssymb}
\usepackage{lineno}
\usepackage[T1]{fontenc}
\usepackage[utf8]{inputenc}
\usepackage{float}
\usepackage{mathpazo}
\usepackage{amsmath}
\usepackage{booktabs}

% Bibliography
\usepackage{natbib}
\bibliographystyle{elsarticle-num-names}

% Highlight
\usepackage{color,soul}
\DeclareRobustCommand{\hlcyan}[1]{{\sethlcolor{cyan}\hl{#1}}}

% Quote
\usepackage[strict]{changepage}
\usepackage{xcolor}
\usepackage{framed}
\definecolor{formalshade}{rgb}{0.95,0.95,1}
\newenvironment{formal}{%
    \def\FrameCommand{%
        \hspace{1pt}%
        {\color{black}\vrule width 2pt}%
        {\color{formalshade}\vrule width 4pt}%
        \colorbox{formalshade}%
    }%
    \MakeFramed{\advance\hsize-\width\FrameRestore}%
    \noindent\hspace{-4.55pt}% disable indenting first paragraph
    \begin{adjustwidth}{}{7pt}%
        \vspace{2pt}\vspace{2pt}%
}
{%
    \vspace{2pt}\end{adjustwidth}\endMakeFramed%
}


% Document
\begin{document}

    \section*{Responses to Reviewers}
    \setcounter{section}{0}

    Dear Editor-in-Chief,

    Thank you for giving us the opportunity to submit a revised draft of our manuscript to Applied Ergonomics.
    We appreciate the time and effort that you, and the reviewers, have dedicated to providing your valuable feedback.
    We have incorporated changes to reflect the suggestions provided by the reviewers and have highlighted the changes within the manuscript.

    Here is a point-by-point response to the reviewers' comments and concerns.
    To simplify proofreading, we have quoted and underlined the text in yellow when an edit has been made to the original text.


    \section{Comments from Reviewer 1}\label{sec:reviewer-1}

    \begin{formal}
        The paper is very well structured;
        the figures are helpful to understand the results.
        The analysed parameters are interesting.
        The authors give supplementary information structured in Appendix.
    \end{formal}

    We would like to thank the reviewer for her/his time and insightful comments that will help us improve the article.

    \begin{formal}
        The parameter of the relative time spent beyond a shear-compression dislocation ratio by sex and mass (section 4.2. \textit{Sex and mass main effects}, figure 7) seems to be less representative for the conclusion.
    \end{formal}
    We agree that this parameter was less relevant to draw the conclusions of this study.
    We considered useful to include it because it links muscle activation, muscle forces and glenohumeral kinematics to the risk of upper limb injury, as stated in the \textit{Kinematic, electromyographic and musculoskeletal evidences of the sex-related differences} section:

    \begin{quote}
        "A greater muscle activation, muscle force and glenohumeral contribution during the handling task will indeed increase the solicitation of passive structures and alter the glenohumeral stability~\citep{Bergmann2007-zj}.
        This high ratio of shear and compression forces imposes high stresses on the stabilizing muscles and thus increases the risk of developing ULMD."
    \end{quote}

    \begin{formal}
        If we focus on the figure, approximatively 35\% (7 men subjects) are considered as outliers.
    \end{formal}

    Figure 7 is based on each of the 196 trials performed by the 20 women (98 trials) and 20 men (98 trials) --- and not on one aggregated data point per participant.
    The seven outliers that you describe are in fact seven trials which correspond to 7\% of the 98 trials performed by men:

    \begin{table}[H]
        \centering
        \caption{Outliers rate in Figure 7}
        \begin{tabular}{lccc}
            \hline
            Sex & 6~kg & 12~kg & Total         \\ \hline
            Women & 1/98 (1.02\%) & 3/98 (3.06\%) & 4/196 (2.04\%) \\
            Men & 7/98 (7.14\%) & 7/98 (7.14\%) & 7/98 (7.14\%) \\ \hline
        \end{tabular}
    \end{table}

    We are sorry for this poor description.
    We added "for each trial" in Figure 7 caption to avoid confusion:

    \begin{quote}
        "Boxplot of the relative time spent beyond a shear-compression dislocation ratio \hl{for each trial} by sex (women in red and men in blue) and mass (6~kg on the left panel and 12~kg on the right panel) with median (horizontal lines), first-third interquartile range (bars), first-third interquartile range multiplied by 1.5 times the interquartile difference [$\textrm{Q1} - 1.5 \times \textrm{IQR}; \textrm{Q2} + 1.5 \times \textrm{IQR}$] (vertical lines).
        Data beyond the end of the whiskers are considered outliers (points)."
    \end{quote}

    \begin{formal}
        Despite this, the authors conclude "in addition, women spent more time beyond a shear-compression dislocation ratio." (section 5. \textit{Discussion}, page 15) and "we have also shown that women spend more time with a high risk of humerus dislocation." (section 5.2. \textit{Kinematic, electromyographic and musculoskeletal evidences of the sex-related differences}).
        More over, using a generic model, it is supposed that the "inter-individual variability is reduced" (section 5.3. \textit{Methodological considerations}).
        With these considerations, the authors are encouraged to be less affirmative on this point.
    \end{formal}

    While the outliers rate remains low, we fully agree and we have nuanced our interpretation of this parameter given the hypotheses it is based on.

    \textit{Abstract}:
    \begin{quote}
        "In addition, women \hl{might} spend more time beyond a shear-compression dislocation ratio."
    \end{quote}

    \textit{Highlights}, point 2:
    \begin{quote}
        "Women generated higher muscle forces and muscle activations when the box was above shoulder level and \hl{might} spent more time beyond a shear-compression dislocation ratio."
    \end{quote}

    \textit{Discussion}:
    \begin{quote}
        "In addition, women \hl{appear} to spend more time beyond a shear-compression dislocation ratio."
    \end{quote}

    \textit{Discussion, section 5.2. Kinematic, electromyographic and musculoskeletal evidences of the sex-related differences}:
    \begin{quote}
        "We have also shown that women \hl{might} spent more time with a high risk of humerus dislocation."
    \end{quote}

    \begin{formal}
        In terms of implications, the authors give different suggestions (section 5.4. \textit{Implications}) that are already known, for example "we recommend a technique that keeps the box closer from the trunk, reduces above shoulder work, makes greater use of the lower limbs and reduces the glenohumeral joint contribution".
        The authors are encouraged to be more ambitious and give some more insights on this point taking into account the results of the study.
    \end{formal}

    We agree that the recommendations could indeed rely more on the main finding of this study which is that women generated higher muscle forces and activations than men when the box was above shoulder level --- at either 6 or 12 kg.
    As such, we added the following sentences (\textit{Implications}, page 21):

    \begin{quote}
        "In our case, these synthetic indicators \hl{reinforced known recommendations} and emphasized the importance of a proper lifting technique on musculoskeletal loads and by extension \textsc{ulmd} prevalence. [\ldots]"
    \end{quote}

    \begin{quote}
        \hl{"Taken together, our studies on kinematics~{\citep{Martinez2019-mm}}, \textsc{emg}~{\citep{Bouffard2019-fd}} and musculoskeletal biomechanics suggest that it is crucial for practitioners to give a special attention to women during above-shoulder work.
        Our indicators show that the biomechanical loads are higher in women compared to men while working above shoulder, which is critical for the development of shoulder injury~{\citep{Van_der_Molen2017-sb}}.
        It would therefore be necessary to adapt workload in women working in such position."}
    \end{quote}

    \section{Comment from Reviewer 2}\label{sec:reviewer-2}

    \begin{formal}
        Thank you for such a nice manuscript.
        You have made a great contribution to the investigation of sex differences in biomechanical loads at work.
    \end{formal}

    We thank the reviewer for her/his kind words and recommendation.

    \begin{formal}
        I only recommend you considering the s\textsc{emg} normalization also as a limitation, as we know that producing a real maximal contraction is something challenging.
        An underestimated MVC (particularly in women) can bias the results.
        If you have taken some special care on that issue I also recommend you reporting it in this study.
    \end{formal}

    While this study does not involve \textsc{emg} --- but rather muscle activations and forces computed by static optimization --- we agree that this is a limitation to be discussed for the results of our previous \textsc{emg} study~\citep{Bouffard2019-fd}, on which we rely to support and explain our results.
    Especially since it was not discussed in the original paper.
    Since normalized \textsc{emg} is the product of the MVC and the measured \textsc{emg}, the MVC is indeed of key importance when comparing samples from different populations --- even more so in women, who may reach different levels of voluntary activation during MVC compared to men~\citep{Ema2018-cd, Inglis2013-uq}.

    In \citet{Bouffard2019-fd}, the normalization procedure was as follows:
    \begin{quote}
        "A series of 12 submaximal voluntary contractions were performed to validate the electrode placement.
        Then, the same muscle contractions were performed with verbal encouragement twice at maximal voluntary intensity for normalisation purposes, in line with the recommendations of~\citet{Dal_Maso2016-wh}."
    \end{quote}

    While the "\textit{Methodological considerations}" section is limited to the specific limitations of this study, we added the following sentence in the previous section (\textit{Kinematic, electromyographic and musculoskeletal evidences of the sex-related differences}, page 17) when interpreting~\citet{Bouffard2019-fd} results:

    \begin{quote}
        \citet{Bouffard2019-fd} also reported higher muscle activation in women, particularly for prime movers.
        \hl{While both studies have specific limitations, they both point to similar conclusions.}
    \end{quote}

    The muscle activations reported in this study are also expressed in \%MVC\@.
    They actually depend on the muscle parameters implemented in the generic model we used.
    Although we tried to adjust the muscle parameters, this is an inherent limitation of using a generic model and one of the main limitations of our study, as stated in the "\textit{Methodological considerations}" section:

    \begin{quote}
        "In addition, the use of a generic model does not allow the musculoskeletal parameters to be personalized for each participant.
        As a result, inter-individual variability is reduced and the activation or strength of certain muscles may be overestimated."
    \end{quote}

    \bibliography{ref}

\end{document}